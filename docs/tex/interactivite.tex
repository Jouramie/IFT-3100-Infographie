\chapter{Interactivité}
\label{s:interactivite}

L’option «Dessiner Primitive» permet d’ajouter une primitive dans la scène. L’utilisateur peut choisir les valeurs pour la couleur du fond de la scène. \\
L’application permet d’importer des modèles et de les représenter en 3D. Il est aussi possible d’exporter à l’aide de l’option Exportation en image. \\
L’option «Type» permet à l’utilisateur de choisir le type de primitive voulu soit en 2D, soit en 3D. \\
Pour les primitives en 3D il est possible de choisir la position en X, Y et Z. La taille est définie par la hauteur, largeur et profondeur.\\
Pour les primitives en 2D il est possible de choisir la position en X et Y.  La taille est définie par la hauteur et largeur.\\
L’utilisateur peut également modifier la couleur de remplissage des primitives choisi, la couleur de la bordure et l’épaisseur des traits.\\ 
L’option «Parametre de la camera» permet de définir tous les paramètres de la caméra de l’application. \\
L’option «Traitement d’image» permet de soit brouiller, inverser ou dilater les éléments de la scène.  \\
Finalement sur le gauche de l’application il est possible de modifier les valeurs de la translation, de la rotation et de la proportion.\\
